\documentclass{article}
\usepackage{amsmath,amssymb,amsthm,latexsym,paralist}

\theoremstyle{definition}
\newtheorem{problem}{Problem}
\newtheorem*{solution}{Solution}
\newtheorem*{resources}{Resources}

\newcommand{\name}[1]{\noindent\textbf{Name: #1}}
\newcommand{\honor}{\noindent On my honor, as an Aggie, I have neither
  given nor received any unauthorized aid on any portion of the
  academic work included in this assignment. Furthermore, I have
  disclosed all resources (people, books, web sites, etc.) that have
  been used to prepare this homework. The solutions given in this
  homework are my own work.\\[1ex]
 \textbf{Signature:} \underline{\hspace*{5cm}} }

 \newcommand{\checklist}{\noindent\textbf{Checklist:}
\begin{compactitem}[$\Box$] 
\item Did you add your name? 
\item Did you disclose all resources that you have used? \\
(This includes all people, books, websites, etc. that you have consulted)
\item Did you sign that you followed the Aggie honor code? 
\item Did you solve all problems? 
\item Did you submit the pdf file of your homework?
\end{compactitem}
}



\newcommand{\problemset}[1]{\begin{center}\textbf{Problem Set
      #1}\end{center}}
\newcommand{\duedate}[2]{\begin{quote}\textbf{Due dates:} Electronic
    submission of the pdf file of this homework is due on
    \textbf{#1} on canvas. \end{quote} }

\newcommand{\N}{\mathbf{N}}
\newcommand{\R}{\mathbf{R}}
\newcommand{\Z}{\mathbf{Z}}


\begin{document}

\problemset{4}
\duedate{2/14/2025 before 11.59pm}{}
\name{ (put your name here)}
\begin{resources} (All people, books, articles, web pages, etc. that
  have been consulted when producing your answers to this homework)
\end{resources}
\honor

\newpage
Make sure that you describe all solutions in your own words. Typeset
your solutions in \LaTeX. Read
chapter 30 on ``Polynomials and the FFT'' and chapter 15 on ``Greedy
Algorithms'' in our textbook. 

\begin{problem}[20 points]
The polynomial $A(x) = 1+x+x^2$ can be represented by the vector
$(1,1,1,0)^t$. (a) Use a $4\times 4$ DFT and transform this
vector. (b) Explicitly describe the resulting vector in terms of a
vector of the form $(A_(x_0), A(x_1), A(x_2), A(x_3))^t$ for some complex
numbers $x_0, x_1, x_2,$ and $x_3$. Make sure that you verify your
result. 
\end{problem}
\begin{solution}
\end{solution}

\begin{problem} (20 points) Let $\omega$ be a primitive $n$th root of unity. 
The fast Fourier transform implements the multiplication with
  the matrix 
$$ F = (\omega^{ij})_{i,j\in [0..n-1]}.$$
Show that the inverse of the matrix $F$ is given by 
$$ F^{-1} = \frac{1}{n}  (\omega^{-jk})_{j,k\in [0..n-1]}$$
[Hint: $x^n-1= (x-1)(x^{n-1}+\cdots + x + 1),$ so any power
$\omega^\ell\neq 1$  must be a root of $x^{n-1}+\cdots + x + 1$.  ]  
Thus, the inverse FFT, called IFFT, is nothing but the FFT using
$\omega^{-1}$ instead of $\omega$, and multiplying the result with
$1/n$. 
\end{problem}
\begin{solution}
\end{solution}

\begin{problem} (20 points) 
Describe in your own words how to do a polynomial multiplication using the FFT and
  IFFT for polynomials $A(x)$ and $B(x)$ of degree $\le n-1$. Make
  sure that you describe the length of the FFT and IFFT needed for
  this task. Be concise and precise. Illustrate how to multiply the
  polynomials $A(x) = x^2+2x+1$ and $B(x)=x^3+2x^2+1$ using this
  approach. 
\end{problem}
\begin{solution}
\end{solution}

\begin{problem} (20 points) 
How can you modify the polynomial multiplication algorithm based
  on FFT and IFFT to do multiplication of long integers in base 10?
  Make sure that you take care of carries in a proper way. Write your
  algorithm in pseudocode and give a brief explanation. 
\end{problem}
\begin{solution}
\end{solution}




\begin{problem}[20 points]
  Describe an efficient algorithm that, given a set
  $$\{ x_1, x_2, \ldots, x_n\}$$ of $n$ points on the real line,
  determines the smallest set of unit-length closed intervals that
  contains all of the given points. Argue that your algorithm is
  correct.

  {\small \textbf{Why do we care?} This is nice opportunity to design
    a simple greedy
    algorithm. Arguing the correctness of a greedy algorithm is
    essential, and this is not too difficult to do in this case.} 
  
\end{problem}
\begin{solution}
\end{solution}





\goodbreak
\checklist
\end{document}
