\documentclass{article}
\usepackage{amsmath,amssymb,amsthm,latexsym,paralist,color,url}
\usepackage[margin=1.5in]{geometry}

\theoremstyle{definition}
\newtheorem{problem}{Problem}
\newtheorem*{solution}{Solution}
\newtheorem*{resources}{Resources}

\newcommand{\name}[1]{\noindent\textbf{Name: #1}}
\newcommand{\honor}{\noindent On my honor, as an Aggie, I have neither
  given nor received any unauthorized aid on any portion of the
  academic work included in this assignment. Furthermore, I have
  disclosed all resources (people, books, web sites, etc.) that have
  been used to prepare this homework. \\[1ex]
 \textbf{Signature:} \underline{\hspace*{5cm}} }

\newcommand{\checklist}{\noindent\textbf{Checklist:}
\begin{compactitem}[$\Box$] 
\item Did you add your name? 
\item Did you disclose all resources that you have used? \\
(This includes all people, books, websites, etc. that you have consulted)
\item Did you sign that you followed the Aggie honor code? 
\item Did you solve all problems? 
\item Did you submit the pdf file
  of your homework?
\end{compactitem}
}

\newcommand{\problemset}[1]{\begin{center}\textbf{Problem Set
      #1}\end{center}}
\newcommand{\duedate}[2]{\begin{quote}\textbf{Due dates:} Electronic
    submission of this homework is due on \textbf{#1} on canvas.\end{quote} }

\newcommand{\N}{\mathbf{N}}
\newcommand{\R}{\mathbf{R}}
\newcommand{\Z}{\mathbf{Z}}

\newcommand{\bl}[1]{\color{blue}#1\color{black}}

\begin{document}
\problemset{8}
\duedate{Friday 4/4/2025 before 11:59pm}{4/4/2025}
\name{ (put your name here)}
\begin{resources} (All people, books, articles, web pages, etc. that
  have been consulted when producing your answers to this homework)
\end{resources}
\honor
\newpage

This homework needs to be typeset in LaTeX to receive any credit. All
answers need to be formulated in your own words. 

You need to gain some familiarity with conditional probabilities, as
they figure prominently in many arguments about randomized
algorithms. The next two exercises give some opportunity to explore
conditional probabilities. 

\begin{problem}[20 points] 
Let $B_1, B_2, \ldots, B_t$ denote a partition of the sample space
$\Omega$. 
\begin{compactenum}[(a)]
\item Prove that $\Pr[A] = \sum_{k=1}^t \Pr[A\mid B_k] \Pr[B_k]$. 
\item Deduce that $\Pr[A] \le \max_{1\le k\le t} \Pr[A \mid B_k].$
\end{compactenum}
\end{problem}
\begin{solution}
\end{solution}


\begin{problem}[20 points] 
Consider an experiment, where you toss two fair coins. 
Give examples of events where (a) $\Pr[A_1 \mid B_1] < \Pr[A_1]$, 
(b) $\Pr[A_2 \mid B_2] =  \Pr[A_2]$, and (c) $\Pr[A_3 \mid B_3] > \Pr[A_3]$. Make
sure that your proofs are complete and self-contained.
\end{problem}
\begin{solution}
\end{solution}

\begin{problem}[20 points] 
  Research the Schwartz-Zippel Randomized Polynomial Identity Test.
  (a) Explain the method. (b) Explain why randomized algorithms are
  better suited for this problem than deterministic algorithms. \par

  Hint: To get started, you can read: \\ \small
  \url{https://web.stanford.edu/class/archive/cs/cs265/cs265.1212/Lectures/Lecture1/l1.pdf}
  \end{problem}
\begin{solution}
\end{solution}

\begin{problem}[20 points] 
  There may be several different min-cut sets in a graph. Using the
  analysis of the randomized min-cut algorithm, argue that there can
  be at most $n(n - 1)/2$ distinct min-cut sets.
\end{problem}
\begin{solution}
\end{solution}

\begin{problem}[20 points] 
A popular choice for pivot selection in Quicksort is the median of
three randomly selected elements. Approximate the probability of
obtaining at worst an $a$-to-$(1-a)$ split in the partition (assuming
that $a$ is a real number in the range $0<a<1/2$). 

\noindent{}[Hint: Suppose that the median-of-three is the $m$-th smallest element
of the array. Then it gives at worst an $a$-to-$(1-a)$ split if and
only if $an \le m\le (1-a)n$. Now count how
many sets of three elements can lead to the the pivot
(= median-of-three) being the $m$-th smallest element. ]

\end{problem}
\begin{solution}
\end{solution}









\goodbreak
\checklist
\end{document}
