\documentclass{article}
\usepackage{amsmath,amssymb,amsthm,latexsym,paralist}
\usepackage{algpseudocode}
\usepackage{algorithm}
\usepackage{algorithmic}
\usepackage[ruled,vlined]{algorithm2e}

\theoremstyle{definition}
\newtheorem{problem}{Problem}
\newtheorem*{solution}{Solution}
\newtheorem*{resources}{Resources}

\newcommand{\name}[1]{\noindent\textbf{Name: #1}}
\newcommand{\honor}{\noindent On my honor, as an Aggie, I have neither
  given nor received any unauthorized aid on any portion of the
  academic work included in this assignment. Furthermore, I have
  disclosed all resources (people, books, web sites, etc.) that have
  been used to prepare this homework. \\[1ex]
 \textbf{Signature: Chayce Leonard} \underline{\hspace*{5cm}} }

\newcommand{\checklist}{\noindent\textbf{Checklist:}
\begin{compactitem}[$\Box$] 
\item Did you add your name? 
\item Did you disclose all resources that you have used? \\
(This includes all people, books, websites, etc. that you have consulted)
\item Did you sign that you followed the Aggie honor code? 
\item Did you solve all problems? 
\item Did you submit the pdf file of your homework?
\end{compactitem}
}



\newcommand{\problemset}[1]{\begin{center}\textbf{Problem Set
      #1}\end{center}}
\newcommand{\duedate}[1]{\begin{quote}\textbf{Due date:} Electronic
    submission of the pdf file of this homework is due on
    \textbf{#1} on canvas. \end{quote} }

\newcommand{\N}{\mathbf{N}}
\newcommand{\R}{\mathbf{R}}
\newcommand{\Z}{\mathbf{Z}}


\begin{document}
\problemset{5}
\duedate{2/23/2025 before 11:59pm}
\name{ Chayce Leonard}
\begin{resources}
Class Textbook \\
Wikipedia \\
GeeksForGeeks (Dynamic Programming Article)


\end{resources}
\honor

\newpage
Make sure that you describe all solutions in your own
words. Typesetting in \LaTeX{} is required. Read
chapters 14 and 15 in our textbook. 

\begin{problem}[20 points]
Determine an LCS of $X = \langle R,H,U,B,A,R,B\rangle$ and
  $Y = \langle S,T,R,A,W,B,E,R,R,Y \rangle$ using the dynamic programming
  algorithm that was discussed in class.  Make sure that you explain your
  answer step-by-step, in detail, rather than just giving an LCS.
  [Make sure that you list $X$ vertically, and $Y$ horizontally when
  constructing the table. If $x_i\neq y_j$, and $c[i-1,j] = c[i,j-1]$,
  choose $c[i-1,j]$. Explain row-by-row how the table is constructed.] 
\end{problem}

{\small \textbf{Why do we care?} For every algorithm, you should make
    sure that you can work it out step-by-step.} 
\begin{solution}
To find the Longest Common Subsequence (LCS) of $X = \langle R,H,U,B,A,R,B\rangle$ and $Y = \langle S,T,R,A,W,B,E,R,R,Y \rangle$, we'll use the dynamic programming algorithm discussed in class. We'll construct a table $c[i,j]$ where $i$ represents the index in $X$ and $j$ represents the index in $Y$. The value in each cell $c[i,j]$ will represent the length of the LCS for the prefixes $X[1..i]$ and $Y[1..j]$ [[4]].

\begin{algorithm}
\caption{LCS Dynamic Programming Algorithm}
\begin{algorithmic}[1]
\State Initialize table $c[0..m,0..n]$ with 0s, where $m = |X|$ and $n = |Y|$
\For{$i = 1$ to $m$}
    \For{$j = 1$ to $n$}
        \If{$x_i = y_j$}
            \State $c[i,j] \gets c[i-1,j-1] + 1$
        \Else
            \State $c[i,j] \gets \max(c[i-1,j], c[i,j-1])$
        \EndIf
    \EndFor
\EndFor
\end{algorithmic}
\end{algorithm}

Now, let's construct the table step-by-step, explaining each row:

\begin{center}
\begin{tabular}{|c|c|c|c|c|c|c|c|c|c|c|c|}
\hline
& & S & T & R & A & W & B & E & R & R & Y \\
\hline
& 0 & 0 & 0 & 0 & 0 & 0 & 0 & 0 & 0 & 0 & 0 \\
\hline
R & 0 & 0 & 0 & 1 & 1 & 1 & 1 & 1 & 1 & 1 & 1 \\
\hline
H & 0 & 0 & 0 & 1 & 1 & 1 & 1 & 1 & 1 & 1 & 1 \\
\hline
U & 0 & 0 & 0 & 1 & 1 & 1 & 1 & 1 & 1 & 1 & 1 \\
\hline
B & 0 & 0 & 0 & 1 & 1 & 1 & 2 & 2 & 2 & 2 & 2 \\
\hline
A & 0 & 0 & 0 & 1 & 2 & 2 & 2 & 2 & 2 & 2 & 2 \\
\hline
R & 0 & 0 & 0 & 2 & 2 & 2 & 2 & 2 & 3 & 3 & 3 \\
\hline
B & 0 & 0 & 0 & 2 & 2 & 2 & 3 & 3 & 3 & 3 & 3 \\
\hline
\end{tabular}
\end{center}

Let's explain the construction of each row:

\begin{enumerate}
    \item \textbf{Row 0 (Initialization):} We initialize the first row and column with 0s, as the LCS of any string with an empty string is 0 [[5]].
    
    \item \textbf{Row 1 (R):} We compare 'R' with each character in Y. When we reach the 'R' in Y (column 3), we get a match, so we add 1 to the value in the cell diagonally up and left (0 + 1 = 1). This value propagates to the right as it's the maximum.
    
    \item \textbf{Row 2 (H):} 'H' doesn't match any character in Y, so we take the maximum of the cell above or to the left for each position. The row remains the same as the previous row.
    
    \item \textbf{Row 3 (U):} Similar to 'H', 'U' doesn't match any character, so this row is identical to the previous one.
    
    \item \textbf{Row 4 (B):} 'B' matches with 'B' in Y (column 6). At this position, we add 1 to the value diagonally up and left (1 + 1 = 2). This new maximum propagates to the right.
    
    \item \textbf{Row 5 (A):} 'A' matches with 'A' in Y (column 4). We get a new maximum of 2 at this position, which continues to the right.
    
    \item \textbf{Row 6 (R):} 'R' matches twice in Y (columns 3 and 9). The first match doesn't increase the maximum, but the second match does, giving us a new maximum of 3 which propagates to the right.
    
    \item \textbf{Row 7 (B):} 'B' matches with 'B' in Y (column 6), increasing the maximum to 3, which continues to the end of the row.
\end{enumerate}

Now that we have constructed the table, we can find the LCS by backtracking from the bottom-right cell $(7,10)$ [[6]]:

\begin{itemize}
    \item Start at $c[7,10] = 3$
    \item Move diagonally up-left when $x_i = y_j$, otherwise move up or left to the larger value
    \item The path: $(7,10) \rightarrow (6,9) \rightarrow (5,4) \rightarrow (4,6)$
    \item Characters on this path: B, R, A, B
\end{itemize}

Therefore, an LCS of X and Y is $\langle B,R,A,B \rangle$.

Note: There might be multiple valid LCSs of the same maximum length. The one we found is determined by our choice to prefer $c[i-1,j]$ when $x_i \neq y_j$ and $c[i-1,j] = c[i,j-1]$, as specified in the problem statement.
\end{solution}
  

\begin{problem}[20 points]
  Find an optimal parenthesization of a matrix-chain product whose
  sequence of dimensions is $\langle 5, 10, 3, 12, 5, 50, 6\rangle$. 
  Explain how you found the solution. [Hint: You might find it
  worthwhile to implement the dynamic programming algorithm, and print
  the relevant tables to aid in your explanations. However, you should
  make sure that you are able to solve it by hand as well.] 
\end{problem}
\begin{solution}
\end{solution}

\begin{problem}[20 points]
  Use a proof by induction to show that the solution to the
  recurrence
  $$ P(n) = 
  \begin{cases}
    1& \text{if $n=1$,} \\
    \sum_{k=1}^{n-1} P(k)P(n-k) & \text{if $n\ge 2$.}
  \end{cases}
  $$
    is $\Omega(2^n)$. 
  \end{problem}

{\small \textbf{Why do we care?} This is a simple lower bound on the
  number of parenthesizations of a chain of $n$ matrices. It will
  remind you about the true meaning of $\Omega(2^n)$. The result serves as a
  reminder why a brute-force solution is not attractive.} 
  
\begin{solution}
\end{solution}

\begin{problem}[20 points]
Let $R(i,j)$ be the number of times that table entry $m[i,j]$ is
referenced while computing other table entries in a call of
MATRIX-CHAIN-ORDER, see [CLRS, Section 14.2].  Show that the total number of references for the
entire table is
$$ \sum_{i=1}^n\sum_{j=i}^n R(i,j) = \frac{n^3-n}{3}.$$
[Hint: Re-read the definition of $R(i,j)$ a couple of times.] 
\end{problem}

{\small \textbf{Why do we care?} This is a key statistics for the
  run-time of the dynamic-programming solution. The manipulation of such sums
  is an essential skill that you need to train.} 

\begin{solution}
\end{solution}

\begin{problem}[20 points]
  Give an $O(n^2)$-time algorithm to find the longest monotonically
  increasing subsequence of a sequence of n numbers.
\end{problem}

{\small \textbf{Why do we care?} You should learn how to apply the
  algorithms from the lecture. This is a good opportunity to hone
  your problem solving skills. Make sure that you solve it yourself
  without any help!} 

\begin{solution}
\end{solution}



Discussions on canvas are always encouraged, especially to clarify
concepts that were introduced in the lecture. However, discussions of
homework problems on canvas should not contain spoilers. It is okay to
ask for clarifications concerning homework questions if needed. Make
sure that you write the solutions in your own words. 


\medskip



\goodbreak
\checklist
\end{document}
